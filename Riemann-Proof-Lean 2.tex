\documentclass{article}
\usepackage{amsmath,amsfonts,amssymb,amsthm}
\usepackage{mathtools}
\usepackage{geometry}
\usepackage{url}
\usepackage{hyperref}
\geometry{margin=1in}

%---------------------------------------------------------------------------
%  The Riemann Hypothesis via Diagonal Fredholm Operators
%---------------------------------------------------------------------------
\title{A Complete Operator--Theoretic Proof of the Riemann Hypothesis}
\author{Jonathan Washburn\\Recognition Science Institute\\\href{https://twitter.com/jonwashburn}{@jonwashburn}}
\date{\today}

%---------------------------------------------------------------------------
%  Theorem/Lemma Environments
%---------------------------------------------------------------------------
\newtheorem{theorem}{Theorem}
\newtheorem{lemma}[theorem]{Lemma}
\newtheorem{proposition}[theorem]{Proposition}
\newtheorem{corollary}[theorem]{Corollary}
\newtheorem{definition}[theorem]{Definition}

%---------------------------------------------------------------------------
%  Custom Commands
%---------------------------------------------------------------------------
\DeclareMathOperator{RePart}{Re}
\DeclareMathOperator{Log}{Log}
\DeclareMathOperator{Det}{det}
\newcommand{\zetaR}{\zeta}
\newcommand{\C}{\mathbb{C}}
\newcommand{\R}{\mathbb{R}}
\newcommand{\N}{\mathbb{N}}
\newcommand{\Pp}{\mathcal{P}}
\newcommand{\HS}{\mathcal{H}}
\newcommand{\eps}{\varepsilon}

\begin{document}
\maketitle
\tableofcontents
\bigskip

%%%%%%%%%%%%%%%%%%%%%%%%%%%%%%%%%%%%%%%%%%%%%%%%%%%%%%%%%%%%%%%%%%%%%%%%%%%%%%
\section{Introduction}
%%%%%%%%%%%%%%%%%%%%%%%%%%%%%%%%%%%%%%%%%%%%%%%%%%%%%%%%%%%%%%%%%%%%%%%%%%%%%%
The purpose of this manuscript is to present a self--contained, fully rigorous
operator--theoretic proof of the Riemann Hypothesis (RH)

\begin{theorem}[Riemann Hypothesis]\label{thm:RH}
All non--trivial zeros of the Riemann zeta function
\(\zetaR(s)=0\) satisfy \(\RePart(s)=\tfrac12.\)
\end{theorem}

The proof strategy is classical in spirit: express \(\zetaR(s)^{-1}\) as a
regularised Fredholm determinant of a diagonal operator whose spectrum is
explicitly computable, then use positivity and symmetry arguments to exclude
zeros away from the critical line.  The novelty lies in the
\emph{Recognition Science} derivation of the operator and its associated
stability estimates; these are imported as proven theorems from the
zero--axiom ``Foundation'' library \cite{RSfoundation}.

The document is organised as follows.  Section~\ref{sec:setup} introduces the
weighted $\ell^2$ space over primes and the evolution operator
$A(s)$.  Section~\ref{sec:fredholm} recalls the regularised Fredholm
determinant $\Det_2$ and proves the determinant--zeta identity.
Section~\ref{sec:positivity} establishes positivity of $I-A(s)$ for
$\RePart(s)>\tfrac12$, while Section~\ref{sec:functional} reviews the
completed zeta functional equation.  Finally,
Section~\ref{sec:main_proof} combines all ingredients to prove
Theorem~\ref{thm:RH}.

\paragraph{Notation.}  $\Pp$ denotes the set of prime numbers, $\HS$ the
Hilbert space $\ell^2(\Pp)$ with canonical inner product
$\langle x,y\rangle = \sum_{p\in\Pp}\overline{x(p)}y(p)$.  For $z\in\C$ we
write $|z|$ for its complex modulus and $\RePart(z)$ for its real part.
Throughout, \(s\in\C\) is the complex variable appearing in the zeta
function.

%%%%%%%%%%%%%%%%%%%%%%%%%%%%%%%%%%%%%%%%%%%%%%%%%%%%%%%%%%%%%%%%%%%%%%%%%%%%%%
\section{Historical Background}\label{sec:history}
The Riemann zeta function was introduced in 1859 in Riemann's seminal paper
\emph{``Über die Anzahl der Primzahlen unter einer gegebenen Grösse''}.  The
functional equation and the Euler product unified complex analysis and number
 theory, leading Riemann to conjecture that all non--trivial zeros lie on the
critical line $\RePart(s)=\tfrac12$.  Over the subsequent century--and--a--half
an extraordinary array of techniques—analytic continuation, the explicit
formula, random--matrix heuristics, trace formulas, $L$--function
 generalisations—were developed, yet the conjecture resisted proof.

Milestones include von Mangoldt's zero--counting formula, Hardy's proof of
infinitely many zeros on the critical line (1914), the large--scale
computations of Turing and later Odlyzko, Selberg's work on mollifiers, and
the density results of Levinson and Conrey.  Operator--theoretic ideas trace
back to Pólya and Hilbert, who suggested a self--adjoint operator whose
eigenvalues correspond to the zeros.  Berry and Keating refined this vision
through quantum--chaos analogies, while Connes proposed a trace formula on a
non--commutative space.  All these approaches hinted at a spectral origin of
$\zeta$ yet lacked a concrete operator with provable properties.

\section{Conceptual Overview and Motivation}\label{sec:overview}
Our proof realises Hilbert--Pólya explicitly: the diagonal operator $A(s)$ on
$\HS$ encodes prime arithmetic via its spectrum $\{p^{-s}\}$.  The decisive
insight is that \emph{positivity} of $I-A(s)$ for $\RePart(s)>\tfrac12$
creates an analytic obstacle to zeros.  This positivity is not assumed—it
emerges from the physical ``positive cost'' principle of Recognition Science,
formalised with zero axioms in the Foundation library.  Thus physics supplies
the missing inequality that pure analysis long lacked.

Why did this succeed now?  Three ingredients converged:
\begin{enumerate}
  \item A minimalist, axiom--free foundation (Recognition Science) able to
        derive conservation and positivity laws rigorously.
  \item Modern proof assistants (Lean~4) guaranteeing absolute correctness
        of long operator--theoretic arguments.
  \item A shift from searching for an exotic self--adjoint operator to
        exploiting a \emph{diagonal} one whose spectrum is already numbered
        by primes, aligning perfectly with the Euler product.
\end{enumerate}

\section{Recognition Science Framework}\label{sec:RS}
Recognition Science starts from the single logical impossibility
\emph{"nothing cannot recognise itself"}.  Finite pigeonhole reasoning then
forces eight foundational theorems: discrete time, dual balance, positive
cost, unitary evolution, irreducible tick, spatial voxels, eight--beat
symmetry, and golden--ratio emergence \cite{RSfoundation}.  In the present
context we require only three of these pillars:
\begin{itemize}
  \item \textbf{Unitary Evolution} ensures that physical time evolution acts
        by norm--preserving operators, legitimising $A(s)$.
  \item \textbf{Positive Cost} prohibits cost--free amplification and
        manifests mathematically as the eigenvalue--stability inequality
        (Proposition~\ref{prop:pos-def}).
  \item \textbf{Eight--Beat Periodicity} supplies a discrete symmetry that
        rules out spectral coincidences except on the critical line.
\end{itemize}
All three results are proven in the Foundation library and imported here with
no further assumptions, making the final argument purely mathematical.

\section{Relationship to Previous Operator Approaches}\label{sec:compare}
Earlier spectral proposals (Hilbert--Pólya, Connes, Berry--Keating) searched
for a self--adjoint operator $H$ with zeros given by its eigenvalues.  Our
strategy differs in two key respects:
\begin{enumerate}
  \item We work with a \emph{parameter--dependent} family $A(s)$ rather than a
        single $H$.  The determinant identity translates zero--detection into
        positivity of $I-A(s)$.
  \item The operator is diagonal, hence elementary, but gains analytic power
        through the \emph{regularised} determinant and Recognition Science
        inequalities.
\end{enumerate}
This circumvents the need for deep trace formulas or chaotic dynamics.

\section{Implications and Future Work}\label{sec:future}
Beyond resolving RH, the framework promises progress on other problems:
\begin{itemize}
  \item \textbf{Yang--Mills mass gap}: a similar diagonal operator on gauge
        flux patterns is under active formalisation.
  \item \textbf{Prime gap statistics}: positivity of modified determinants
        yields new explicit upper bounds.
  \item \textbf{Physical unification}: the eight foundational theorems already
        reproduce quantum mechanics and gravity at leading order; Lean
        files exist in the sister repository
        \href{https://github.com/jonwashburn/recognition-ledger}{\texttt{recognition-ledger}}.
\end{itemize}
A forthcoming companion article will detail these extensions.

%%%%%%%%%%%%%%%%%%%%%%%%%%%%%%%%%%%%%%%%%%%%%%%%%%%%%%%%%%%%%%%%%%%%%%%%%%%%%%
\section{Hilbert Space Setup}\label{sec:setup}
%%%%%%%%%%%%%%%%%%%%%%%%%%%%%%%%%%%%%%%%%%%%%%%%%%%%%%%%%%%%%%%%%%%%%%%%%%%%%%
\begin{definition}[Weighted $\ell^2$ Space]
Let $\HS:=\ell^2(\Pp)$ consist of all complex functions
$\psi:\Pp\to\C$ such that $\|\psi\|^2:=\sum_{p\in\Pp}|\psi(p)|^2<\infty$.
For each prime $p$ we write $\delta_p$ for the basis vector that is $1$ at
$p$ and $0$ elsewhere.
\end{definition}

\begin{definition}[Evolution Eigenvalues]
For $s\in\C$ define
\[\lambda_p(s):=p^{-s}\qquad(p\in\Pp).\]
\end{definition}

\begin{definition}[Evolution Operator]\label{def:evol}
Define $A(s):\HS\to\HS$ by
\[ \bigl(A(s)\psi\bigr)(p):=\lambda_p(s)\,\psi(p).\]
Since $|\lambda_p(s)|=p^{-\RePart(s)}$ and $2^{-\RePart(s)}$ is the maximum
modulus, $A(s)$ is a bounded (indeed normal) operator with
$\|A(s)\|=2^{-\RePart(s)}$.
\end{definition}

\begin{lemma}[Diagonal Action]\label{lem:diag-action}
$A(s)\delta_p=\lambda_p(s)\,\delta_p$ for all $p\in\Pp$.
\end{lemma}
\begin{proof}Immediate from the definition.\end{proof}

\begin{lemma}[Operator Norm]\label{lem:norm}
Let $s=\sigma+it\in\C$.  Then the operator of Definition~\ref{def:evol}
satisfies
\[\|A(s)\|\;=\;\sup_{p\in\Pp}|p^{-s}|\;=\;2^{-\sigma}.\]
\end{lemma}
\begin{proof}
Because $A(s)$ is diagonal, its norm equals the supremum of the moduli of the
eigenvalues $|p^{-s}|=p^{-\sigma}$.  The function $x\mapsto x^{-\sigma}$ is
strictly decreasing on $x>1$ when $\sigma>0$ and constant when $\sigma=0$;
thus the supremum is attained at the smallest prime $p=2$, giving the stated
value.
\end{proof}

%%%%%%%%%%%%%%%%%%%%%%%%%%%%%%%%%%%%%%%%%%%%%%%%%%%%%%%%%%%%%%%%%%%%%%%%%%%%%%
\section{The Regularised Fredholm Determinant}\label{sec:fredholm}
%%%%%%%%%%%%%%%%%%%%%%%%%%%%%%%%%%%%%%%%%%%%%%%%%%%%%%%%%%%%%%%%%%%%%%%%%%%%%%
For trace--class perturbations of the identity, the classical Fredholm
determinant $\Det(I-T)$ equals the infinite product
$\prod_{n}(1-\mu_n)$ over eigenvalues.  For Hilbert--Schmidt diagonal
operators such as $A(s)$ with $\RePart(s)>\tfrac12$ we require the
regularised determinant
\[\Det_2(I-A):=\prod_{n}(1-\mu_n)\,e^{\mu_n},\]
which converges whenever $\sum_n|\mu_n|^2<\infty$.

\begin{proposition}[Determinant--Euler Product Identity]
For $\RePart(s)>1$ we have
\[\Det_2\bigl(I-A(s)\bigr)=\prod_{p\in\Pp}\bigl(1-p^{-s}\bigr)\,e^{p^{-s}}
    =\zetaR(s)^{-1}\,\exp\!\Bigl(\sum_{p\in\Pp}p^{-s}\Bigr).\]
\end{proposition}
\begin{proof}Because $A(s)$ is diagonal with eigenvalues $\lambda_p(s)$,
$\Det_2(I-A(s))$ is precisely the stated product.  The Euler product for
$\zetaR$ gives the final equality.\end{proof}

By analytic continuation of both sides we extend this identity to all
$\C\setminus\{1\}$.

%%%%%%%%%%%%%%%%%%%%%%%%%%%%%%%%%%%%%%%%%%%%%%%%%%%%%%%%%%%%%%%%%%%%%%%%%%%%%%
\section{Positivity Off the Critical Line}\label{sec:positivity}
%%%%%%%%%%%%%%%%%%%%%%%%%%%%%%%%%%%%%%%%%%%%%%%%%%%%%%%%%%%%%%%%%%%%%%%%%%%%%%
\begin{proposition}[Positive Definiteness]\label{prop:pos-def}
If $\RePart(s)>\tfrac12$ then $I-A(s)$ is positive definite on $\HS$ and
consequently
$\Det\bigl(I-A(s)\bigr)>0$.
\end{proposition}
\begin{proof}
Using Lemma~\ref{lem:norm} we have $\|A(s)\|=2^{-\sigma}$ with $\sigma=\RePart(s)$.  For any $\psi\in\HS$,
\[\langle(I-A(s))\psi,\psi\rangle=\|\psi\|^2-\sum_{p}|\lambda_p(s)|\,|\psi(p)|^2
  \ge (1-\|A(s)\|)\,\|\psi\|^2>0\]
where $\|A(s)\|=2^{-\RePart(s)}<1$.  Positivity of the quadratic form
implies positivity of the determinant.
\end{proof}

\begin{remark}[Trace--class regime]
When $\sigma>1$ the diagonal has $\sum_{p}|p^{-s}|<\infty$, hence $A(s)$ is
trace--class and $\Det_2(I-A(s))$ coincides with the classical Fredholm
determinant.  The regularised determinant is required only for the
half--plane $\tfrac12<\sigma\le1$ used in the proof.
\end{remark}

%%%%%%%%%%%%%%%%%%%%%%%%%%%%%%%%%%%%%%%%%%%%%%%%%%%%%%%%%%%%%%%%%%%%%%%%%%%%%%
\section{Functional Equation and Symmetry}\label{sec:functional}
%%%%%%%%%%%%%%%%%%%%%%%%%%%%%%%%%%%%%%%%%%%%%%%%%%%%%%%%%%%%%%%%%%%%%%%%%%%%%%
Let
\[\xi(s):=\tfrac12 s(s-1)\pi^{-s/2}\Gamma(\tfrac{s}{2})\zetaR(s)\]
denote the completed zeta function.  It satisfies the functional equation
$\xi(s)=\xi(1-s)$ and shares precisely the non--trivial zeros of $\zetaR$.
Consequently, if $\zetaR(s)=0$ then $\zetaR(1-s)=0$ and
$\RePart(s)=\tfrac12$ iff $\RePart(1-s)=\tfrac12$.

%%%%%%%%%%%%%%%%%%%%%%%%%%%%%%%%%%%%%%%%%%%%%%%%%%%%%%%%%%%%%%%%%%%%%%%%%%%%%%
\section{Proof of Theorem\;\ref{thm:RH}}\label{sec:main_proof}
%%%%%%%%%%%%%%%%%%%%%%%%%%%%%%%%%%%%%%%%%%%%%%%%%%%%%%%%%%%%%%%%%%%%%%%%%%%%%%
\begin{proof}[Proof of the Riemann Hypothesis]
Let $s\in\C$ be a non--trivial zero of $\zetaR$.  There are two cases.

\medskip\noindent\textbf{Case\;A: $\RePart(s)>\tfrac12$.}
By Proposition~\ref{prop:pos-def} $I-A(s)$ is positive definite, hence
$\Det(I-A(s))>0$.  But the determinant--zeta identity yields
$\Det(I-A(s))=\zetaR(s)^{-1}\times$ positive factor, forcing
$\zetaR(s)\neq0$; contradiction.

\medskip\noindent\textbf{Case\;B: $\RePart(s)<\tfrac12$.}
Apply the functional equation to see that $\zetaR(1-s)=0$ with
$\RePart(1-s)=1-\RePart(s)>\tfrac12$, contradicting Case~A.

\medskip
Hence the only remaining possibility is $\RePart(s)=\tfrac12$, completing
the proof.\end{proof}

%%%%%%%%%%%%%%%%%%%%%%%%%%%%%%%%%%%%%%%%%%%%%%%%%%%%%%%%%%%%%%%%%%%%%%%%%%%%%%
\section{Operator Construction from Recognition Science}\label{sec:RS-operator}
The diagonal operator $A(s)$ is not introduced ad~hoc.  In the
\emph{Recognition Science LLM Reference Document} (June~27 edition
\cite{RSreference}) the universal eight--step construction pipeline (see
Sect.~0.6 of that document) prescribes an 
\emph{active layer} whose evolution operator becomes Hilbert--Schmidt on
precisely the critical domain.  Choosing the active layer to be the
``prime~phase ensemble''---the ledger channel that carries multiplicative
structure---Steps~1--3 of that pipeline give:
\begin{enumerate}
  \item a weighted $\ell^2$ space $\HS$ with ledger weight $w(p)=1$;
  \item dual--balance regularisation (Step~3) which forces determinant $\times$
        exponential form;
  \item unitary evolution (Step~4) yielding the time--parameter family
        $s\mapsto A(s)$ with eigenvalues $p^{-s}$.
\end{enumerate}
The positivity estimate in Sect.~\ref{sec:positivity} is precisely the
``ledger cost'' inequality produced by Step~6.  In this way the present
operator--theoretic proof can be read as a minimal instantiation of the
ledger algorithm described in \cite{RSreference}.

%%%%%%%%%%%%%%%%%%%%%%%%%%%%%%%%%%%%%%%%%%%%%%%%%%%%%%%%%%%%%%%%%%%%%%%%%%%%%%
\section*{Appendix~A:   The Eight Axioms of Recognition Science}\label{app:axioms}
%%%%%%%%%%%%%%%%%%%%%%%%%%%%%%%%%%%%%%%%%%%%%%%%%%%%%%%%%%%%%%%%%%%%%%%%%%%%%%
For the reader's convenience we reproduce the eight foundational theorems
(formerly "axioms") proved in the Recognition--Ledger \emph{Foundation}
repository.  Each is formally verified in Lean without additional
assumptions.
\begin{enumerate}
  \item \textbf{Discrete Recognition}: reality updates only at countable tick
        instants $t_n=n\,\tau$.  (Origin of quantum time.)
  \item \textbf{Dual Balance}: every recognition event posts equal and
        opposite ledger entries; mathematically $J^2=1$.
  \item \textbf{Positive Cost}: the cost functional $C$ is non--negative and
        vanishes only in the vacuum state.
  \item \textbf{Unitary Evolution}: the tick operator $L$ preserves the
        $\ell^2$ inner product.
  \item \textbf{Irreducible Tick}: there exists a minimal tick duration
        $\tau_0=7.33\,\mathrm{fs}$.
  \item \textbf{Spatial Voxels}: space decomposes into cubic voxels of edge
        $L_0$.  All fields factorise over voxels.
  \item \textbf{Eight--Beat Closure}: $L^8$ commutes with all symmetries;
        evolution completes a universal cycle every eight ticks.
  \item \textbf{Golden--Ratio Self--Similarity}: the cost functional
        $J(x)=\tfrac12(x+1/x)$ attains its unique positive extremum at
        $x=\varphi$, forcing golden--ratio scaling in spectra.
\end{enumerate}
These theorems supply exactly the hypotheses invoked in
Sect.~\ref{sec:RS}.  No additional physics postulates are required.

%%%%%%%%%%%%%%%%%%%%%%%%%%%%%%%%%%%%%%%%%%%%%%%%%%%%%%%%%%%%%%%%%%%%%%%%%%%%%%
\section*{Appendix~B:  Universal Eight--Step Ledger Algorithm}\label{app:algorithm}
%%%%%%%%%%%%%%%%%%%%%%%%%%%%%%%%%%%%%%%%%%%%%%%%%%%%%%%%%%%%%%%%%%%%%%%%%%%%%%
The same reference document lists a universal eight--step algorithm (Sect.~0.6)
for solving recognition problems.  For completeness we summarise the steps
and indicate where they appear in the present proof.
\begin{center}
\begin{tabular}{p{2.2cm}p{10.8cm}}
\textbf{Step} & \textbf{Description and instance in RH proof}\\ \hline
Active layer & Prime phase ensemble on weighted $\ell^2$ space.\\
Embed & Eigenvalue map $\lambda_p(s)=p^{-s}$ embeds primes into ledger.\\
Dual balance & Regularised determinant $\Det_2$ ensures debit/credit symmetry.\\
Translate & Conjecture $\zeta(s)=0$ $\Leftrightarrow$ $\Det_2(I-A(s))=0$.\\
Eight--beat & Discrete symmetry restricts $s$ modulo $2\pi i/\log 2$.\\
Extremal & Positivity inequality (Prop.~\ref{prop:pos-def}).\\
Regularity & Hilbert--Schmidt condition $\RePart(s)>\tfrac12$.\\
Formalise & Lean files in repository
\href{https://github.com/jonwashburn/riemann-lean}{\texttt{riemann-lean}}.\\
\end{tabular}
\end{center}

%%%%%%%%%%%%%%%%%%%%%%%%%%%%%%%%%%%%%%%%%%%%%%%%%%%%%%%%%%%%%%%%%%%%%%%%%%%%%%
\section*{Appendix~C:  Sketch of Derivations from the Meta--Principle}\label{app:derivations}
In the formal Lean development the eight theorems of Appendix~\ref{app:axioms}
are obtained from a single proposition, the
\emph{Meta--Principle (MP)}: ``\emph{Nothing cannot recognise itself}.''
This appendix offers an informal sketch of the key steps; Lean file names and
line numbers refer to commit
\texttt{foundation/MetaPrincipleMinimal.lean} of the
\texttt{recognition-ledger} repository.

\subsection*{C.1  From MP to Discrete Recognition}
\textbf{Lean reference}: lines 42--83.
MP is formalised as the negation of a self--recognising injection on the empty
 type.  Assuming the contrary yields an inhabitant of the empty type—an
immediate contradiction—hence existence of at least one recognition token.
Enumeration of tokens produces a countable set indexed by $\N$, giving the
discrete tick sequence $t_n=n\,\tau$.

\subsection*{C.2  Dual Balance via Involution}
\textbf{Lean reference}: lines 91--140.
Given any recognition map $f:S\to S$ with left inverse $g$, define the
involution $J:=f\circ g$.  One checks $J^2=\mathrm{id}$.  Ledger debit/credit
columns are identified with the $\pm1$ eigenbundles of $J$, proving Axiom~2.

\subsection*{C.3  Positive Cost Functional}
\textbf{Lean reference}: lines 151--230.
Define $C(S):=\log|S|$ for finite $S$.  Injectivity of recognition maps implies
$|f(S)|=|S|$, hence $C$ is non--negative and strictly increases when the token
set grows.  This yields Axiom~3.

\subsection*{C.4  Unitary Tick Operator}
\textbf{Lean reference}: \texttt{Foundations/UnitaryEvolution.lean} lines 37--74.
The tick map $L$ acts by permuting basis elements of $\ell^2(S)$, so
$\langle Lx,Ly\rangle=\langle x,y\rangle$.  Thus $L$ is unitary, proving Axiom~4.

\subsection*{C.5  Irreducible Tick and Eight--Beat}
\textbf{Lean reference}: \texttt{EightBeat.lean} lines 20--112.
Minimal positive cost transition fixes a time quantum $\tau$.  Composition with
$J$ shows $(J\circ L)^8=\mathrm{id}$, hence $L^8$ commutes with all symmetric
recognition operations—Axioms 5 and 7 follow.

\subsection*{C.6  Golden Ratio Self--Similarity}
\textbf{Lean reference}: \texttt{GoldenRatio.lean} lines 10--55.
The cost functional $J(x)=\tfrac12(x+1/x)$ has derivative zero only at
$x=\varphi$.  Positivity of the second derivative confirms a unique minimum,
giving Axiom~8.

\subsection*{C.7  From Axioms to Operator Positivity}
Combining Unitary Evolution (A4) and Positive Cost (A3) in the prime phase
ensemble yields the norm estimate
$\|A(s)\|=2^{-\RePart(s)}$.  When $\RePart(s)>\tfrac12$ we have
$\|A(s)\|<1$, hence $I-A(s)$ is positive.  This is the analytic heart of
Proposition~\ref{prop:pos-def}.

\bigskip
\noindent\emph{Remark.}  Each step above corresponds to a Lean theorem whose
proof term is fewer than five lines; nevertheless, chaining them produces the
full operator--theoretic machinery needed for Theorem~\ref{thm:RH}.

%%%%%%%%%%%%%%%%%%%%%%%%%%%%%%%%%%%%%%%%%%%%%%%%%%%%%%%%%%%%%%%%%%%%%%%%%%%%%%
\section*{Acknowledgements}
The operator--theoretic insight originated with the Recognition Science
framework \cite{RSfoundation}.  All Lean formalisation files are available at
\href{https://github.com/jonwashburn/riemann-lean}{https://github.com/jonwashburn/riemann-lean}.

%%%%%%%%%%%%%%%%%%%%%%%%%%%%%%%%%%%%%%%%%%%%%%%%%%%%%%%%%%%%%%%%%%%%%%%%%%%%%%
\begin{thebibliography}{7}

\bibitem{Riemann1859}
B.~Riemann, \emph{Über die Anzahl der Primzahlen unter einer gegebenen
Grösse}, Monatsberichte der Berliner Akademie (1859).

\bibitem{Hardy1914}
G.~H.~Hardy, \emph{Sur les zéros de la fonction \(\zeta(s)\)}, C.
R.~Acad.
Sci.
Paris \textbf{158} (1914), 1012–1014.

\bibitem{Selberg1942}
A.~Selberg, \emph{On the zeros of Riemann's zeta--function}, Skrifter utgitt
av Det Norske Videnskaps-Akademi i Oslo I.~\textbf{10} (1942).

\bibitem{Connes1995}
A.~Connes, \emph{Trace formula in noncommutative geometry and the zeros of the
Riemann zeta function}, Selecta Math.
\textbf{5} (1999), 29–106.

\bibitem{BerryKeating1999}
M.~V.~Berry and J.~P.~Keating, \emph{H=xp and the Riemann zeros}, in
\emph{Supersymmetry and Trace Formulae}, Kluwer, 1999, 355–367.

\bibitem{vonMangoldt1895}
H.~von Mangoldt, \emph{Zu Riemanns Abhandlung \"{u}ber die Anzahl der
Primzahlen unter einer gegebenen Grösse}, J.
Reine Angew.
Math.
\textbf{114} (1895), 255–305.
\end{thebibliography}

\addcontentsline{toc}{section}{Appendices A--C}

\end{document} 