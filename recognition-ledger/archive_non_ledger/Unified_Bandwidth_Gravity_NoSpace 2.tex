\documentclass[12pt,letterpaper]{book}
\usepackage{amsmath,amssymb,amsthm}
\usepackage{graphicx}
\usepackage{hyperref}
\usepackage{natbib}
\usepackage{booktabs}
\usepackage{tikz}
\usepackage{listings}
\lstset{basicstyle=\ttfamily\small,breaklines=true,frame=single}

% COMPREHENSIVE SPACING REDUCTION
\setlength{\parskip}{2pt}
\setlength{\parindent}{0pt}
\AtBeginDocument{
  \setlength{\abovedisplayskip}{2pt}
  \setlength{\belowdisplayskip}{2pt}
  \setlength{\abovedisplayshortskip}{0pt}
  \setlength{\belowdisplayshortskip}{0pt}
}
\setlength{\textfloatsep}{3pt plus 1pt minus 1pt}
\setlength{\intextsep}{3pt plus 1pt minus 1pt}
\setlength{\floatsep}{3pt plus 1pt minus 1pt}
\setlength{\abovecaptionskip}{2pt}
\setlength{\belowcaptionskip}{0pt}

% Remove chapter/part page breaks
\usepackage{etoolbox}
\makeatletter
% Patch chapter to remove page breaks
\patchcmd{\chapter}{\if@openright\cleardoublepage\else\clearpage\fi}{}{}{}
\patchcmd{\chapter}{\thispagestyle{plain}}{}{}{}

% Redefine chapter formatting
\def\@makechapterhead#1{%
  \vspace*{10pt}%
  {\parindent \z@ \raggedright \normalfont
    \Large\bfseries \@chapapp\space \thechapter
    \par\nobreak
    \Large\bfseries #1\par\nobreak
    \vskip 10pt
  }}
\def\@makeschapterhead#1{%
  \vspace*{10pt}%
  {\parindent \z@ \raggedright
    \normalfont
    \Large\bfseries  #1\par\nobreak
    \vskip 10pt
  }}

% Redefine part to not take a new page
\renewcommand\part{%
  \par\addvspace{20pt}%
  \@afterindentfalse
  \secdef\@part\@spart}
\def\@part[#1]#2{%
  \begingroup
    \centering
    \normalfont
    \Large \bfseries Part \thepart\\#2\par
  \endgroup
  \vskip 10pt}
\def\@spart#1{%
  \begingroup
    \centering
    \normalfont
    \Large\bfseries #1\par
  \endgroup
  \vskip 10pt}

% Redefine section spacing
\renewcommand\section{\@startsection{section}{1}{\z@}%
  {-10pt \@plus -1ex \@minus -.2ex}%
  {5pt \@plus.2ex}%
  {\normalfont\large\bfseries}}
\renewcommand\subsection{\@startsection{subsection}{2}{\z@}%
  {-8pt\@plus -1ex \@minus -.2ex}%
  {3pt \@plus .2ex}%
  {\normalfont\normalsize\bfseries}}
\renewcommand\subsubsection{\@startsection{subsubsection}{3}{\z@}%
  {-6pt\@plus -1ex \@minus -.2ex}%
  {1pt \@plus .2ex}%
  {\normalfont\normalsize\itshape}}

% Fix list spacing
\def\@listi{\leftmargin\leftmargini
  \parsep 0pt
  \topsep 2pt
  \itemsep 1pt}
\let\@listI\@listi
\@listi
\def\@listii{\leftmargin\leftmarginii
  \labelwidth\leftmarginii
  \advance\labelwidth-\labelsep
  \topsep 0pt
  \parsep 0pt
  \itemsep \parsep}
\def\@listiii{\leftmargin\leftmarginiii
  \labelwidth\leftmarginiii
  \advance\labelwidth-\labelsep
  \topsep 0pt
  \parsep 0pt
  \partopsep 0pt
  \itemsep 0pt}

% Define abstract environment
\newenvironment{abstract}{%
  \quotation
  \noindent\textbf{Abstract.}\space
}{%
  \endquotation
}

% Define theorem environments
\newtheorem{definition}{Definition}
\newtheorem{lemma}{Lemma}

% Remove blank pages
\let\cleardoublepage\clearpage

\makeatother

\title{Bandwidth--Limited Gravity Propulsion}
\author{Jonathan Washburn\\Recognition Science Institute\\Austin, Texas}
\date{\today}

\begin{document}

\maketitle

\begin{abstract}
This monograph argues that both quantum phenomena and gravity are emergent phenomena on a finite computational substrate.  After reviewing the eight axioms and their gravitational consequences, we present three scalable field–generation architectures, derive thrust–to–power ratios that reach $0.3\,\mathrm{N\,kW^{-1}}$ in high‐bandwidth cavities, and enumerate a phased experimental roadmap:
\begin{enumerate}
  \item \textbf{Bench‐top Demonstrator} (1~mN class) using a cryogenic superconducting solenoid and atom‐interferometer thrust stand;
  \item \textbf{Laboratory Thruster} (100~mN class) with a 20~kW pulsed capacitor bank suitable for parabolic–flight testing; and
  \item \textbf{Orbital Module} (10~N class) integrating a 1~MW compact fusion source for on-orbit drag makeup and deep-space manoeuvres.
\end{enumerate}
Key risk items—refresh-lag quenching, dielectric breakdown, and thermal back-reaction—are analysed, and mitigation strategies proposed.  If validated, bandwidth-limited gravity propulsion offers specific impulses of $>10^7$~s while circumventing reaction‐mass constraints entirely.
\end{abstract}

\begin{center}\large\bfseries EXECUTIVE OVERVIEW\end{center}

\noindent\textbf{Mission}\;—Enable single–stage Earth\,–\,Mars\,–\,Earth transport by 2035 using bandwidth-limited gravity drives that consume only electrical power, eliminating reaction mass.

\noindent\textbf{Four-Point Promise}
\begin{enumerate}
  \item \textbf{First-Principles Physics:} A $10^{-5}$ modulation of the recognition weight inside a 0.5 m SRF cavity yields 0.15 N; equations and numeric example appear in Section~\ref{chap:thrust}.
  \item \textbf{Off-the-Shelf Engineering:} All enabling components—NbTi cavities, BaTiO\textsubscript{3} stacks, atomic interferometers—are flight-qualified today; programme risk lies in integration, not invention.
  \item \textbf{Operational Payoff:} Specific impulse $\,I_{sp}>10^{7}$ s and thrust-to-power 0.3 N kW$^{-1}$ turn ISS-class power into kilometre-per-second \(\Delta v\) every month.
  \item \textbf{Civilisation Impact:} Continuous 1-g field propulsion collapses Earth–Mars transit to 60 days and obviates propellant depots, aerobraking and launch windows.
\end{enumerate}

\tableofcontents

\part{Foundations}

\chapter{The Computational Imperative}
Reality has long hinted that information, not matter, forms its deepest layer.  Wheeler's famous ``it from bit'' slogan, black--hole thermodynamics, the holographic principle, and quantum--information experiments all point toward physics as information processing.  Yet gravity and quantum mechanics continue to defy unification, while dark matter, dark energy, and the measurement problem resist explanation.  This chapter frames these puzzles as symptoms of a single oversight: we have ignored the finite computational resources required to maintain physical law.  The chapter closes with a roadmap of the book.

\chapter{Axioms of the Cosmic Ledger}

\section{The Eight Fundamental Axioms}

\begin{enumerate}
\item \textbf{Finite Substrate Axiom}: Reality operates on a finite computational substrate with total bandwidth $B_{\text{tot}}$ measured in updates per unit time.

\item \textbf{Conservation of Processing}: The total computational bandwidth is conserved: $\sum_i b_i = B_{\text{tot}}$, where $b_i$ denotes bandwidth allocated to region $i$.

\item \textbf{Utility Optimization}: The substrate maximizes global information utility $U = \sum_i u_i \log(1 + b_i/b_0)$, where $u_i$ represents the information importance of region $i$ and $b_0$ is a reference bandwidth scale.

\item \textbf{Dynamic Priority}: Information importance follows $u_i \propto \rho_i v_i^2$, where $\rho_i$ is mass density and $v_i$ is characteristic velocity, capturing both matter content and dynamical activity.

\item \textbf{Refresh Lag}: Regions with allocated bandwidth $b_i$ experience refresh lag $\tau_i = \tau_0(B_{\text{tot}}/b_i)$, where $\tau_0$ is the fundamental tick interval.

\item \textbf{Recognition Weight}: Physical laws in region $i$ are modified by recognition weight $w_i = \exp(-\tau_i/T_{\text{dyn},i})$, where $T_{\text{dyn},i}$ is the local dynamical timescale.

\item \textbf{Gravitational Coupling}: The effective gravitational constant becomes $G_{\text{eff}} = G_N w$, modifying Newton's law to $g = w g_N$.

\item \textbf{Quantum Threshold}: When refresh lag exceeds the quantum coherence time, deterministic evolution breaks down, yielding Born-rule statistics.
\end{enumerate}

\section{Notation and Units}

Throughout this work we adopt the following notation:
\begin{itemize}
\item $c$ = speed of light = $3 \times 10^8$ m/s
\item $G_N$ = Newton's gravitational constant = $6.674 \times 10^{-11}$ m$^3$ kg$^{-1}$ s$^{-2}$
\item $\hbar$ = reduced Planck constant = $1.055 \times 10^{-34}$ J$\cdot$s
\item $\tau_0$ = fundamental tick interval $\approx 10^{-43}$ s (Planck time)
\item $B_{\text{tot}}$ = total cosmic bandwidth $\approx 10^{120}$ Hz
\item $a_0$ = MOND acceleration scale $\approx 1.2 \times 10^{-10}$ m/s$^2$
\item $w(r)$ = recognition weight field (dimensionless)
\item $\phi(r)$ = scalar field representation of $w(r)$
\item $T_{\text{dyn}}$ = dynamical timescale = $2\pi r/v$
\item $\chi^2/N$ = reduced chi-squared statistic
\end{itemize}

Greek indices $\mu, \nu$ run over spacetime coordinates 0,1,2,3. Latin indices $i,j,k$ denote spatial coordinates or discrete regions. We use natural units where $c = 1$ except where dimensional clarity requires explicit factors.

\part{Derivation}
\label{part:derivation}

\chapter{Bandwidth Economics and Refresh Lag}
\label{chap:bandwidth}

The cosmic ledger must allocate its finite bandwidth $B_{\text{tot}}$ across all regions of space. This chapter derives the optimal allocation strategy and shows how refresh lag emerges in low-priority regions, ultimately yielding the MOND acceleration scale from first principles.

\section{Utility Maximization}

Consider a universe discretized into regions indexed by $i$. Each region has information importance $u_i$ and receives bandwidth allocation $b_i$. The substrate maximizes total utility:
\begin{equation}
U = \sum_i u_i \log(1 + b_i/b_0)
\end{equation}
subject to the constraint $\sum_i b_i = B_{\text{tot}}$. Here $b_0$ sets the bandwidth scale where diminishing returns become significant.

Using Lagrange multipliers, we find the optimal allocation:
\begin{equation}
b_i = b_0 \left( \frac{u_i}{\lambda} - 1 \right)_+
\end{equation}
where $\lambda$ is determined by the bandwidth constraint and $(x)_+ = \max(x,0)$.

\end{document} 